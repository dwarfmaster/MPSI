\documentclass{article}

\usepackage[utf8]{inputenc}
\usepackage[T1]{fontenc}
\usepackage[french]{babel}
\usepackage[top=2cm,left=2cm,right=3cm,left=3cm]{geometry}
\usepackage{amsmath}
\usepackage{amssymb}
\usepackage{amsthm}
\usepackage{mathrsfs}

\newcommand{\tld}[1]{\widetilde{#1}}

\title{Intégration numérique}
\author{Luc Chabassier}

\begin{document}
\maketitle

Il s'agit de calculer numériquement des intégrales.

\section{Décompostion en quadratures élémentaires}
On a \[\int_a^b f(x)dx = \sum_{i=0}^{n-1} \int_{\alpha_i}^{\alpha_{i+1}} f(x)dx \]

\section{Quadrature élémentaire}
\subsection{Ordre d'une méthode}
Une méthode de quadrature est d'ordre $N$ si la formule approchée est exacte pour tout polynôme de degré $N$ et inexacte pour au moins un polynôme de degré $N+1$.

\subsection{Méthode des rectangles}
\subsubsection{Cas général}
On approxime la fonction $f$ par la fonction constante de valeur : $y = f(\xi_i), \xi_i \in [\alpha_i, \alpha_{i+1}]$.

On a : \begin{equation}
    \int_{\alpha_i}^{\alpha_{i+1}} f(x)dx \approx (\alpha_{i+1} - \alpha_i)f(\xi_i)
\end{equation}

\subsubsection{Méthode des rectangles à gauche}
On prend $\xi_i = \alpha_i$

Si $f$ est croissante, on obtient une minoration de l'intégrale.

Si $f$ est décroissante, on obtient une majoration de l'intégrale.

\subsubsection{Méthode des rectangle à droite}
On prend $\xi_i = \alpha_{i+1}$

Si $f$ est croissante, on obtient une majoration de l'intégrale.

Si $f$ est déccroissante, on obtient une minoration de l'intégrale.

\subsubsection{Méthode du point millieu}
On prend $\xi_i = \frac{\alpha_{i+1} - \alpha_i}{2}$

On n'obtient ni majoration ni minoration, mais la méthode est plus précise.

\subsubsection{Étude de l'ordre}
Les méthodes des rectangles à droite à gauche et à gauche sont des méthodes d'ordre zéro. Si on prend $f(x) = C, C\in\mathbb{R}$. On a, quel que soit le $\xi_i$ choisit :
\[
    \left\{\begin{aligned} &\int_a^b f(c)dx = (b-a)*C \\
                           &\sum_{i=0}^{n-1} \int_{\alpha_i}^{\alpha_{i+1}} f(\xi_i)dx = (b-a)*C
    \end{aligned}\right.
\]

Le calcul fonctionne bien. Cependant, si on prend $f(x) = mx+p, (m,p)\in\mathbb{R}^2$, la méthode des rectangles à droite donne :
\[
    \left\{\begin{aligned}
            & \int_{\alpha_i}^{\alpha_{i+1}} f(x)dx = \left[\frac{1}{2}mx^2+px\right]_{\alpha_i}^{\alpha_{i+1}} = \frac{1}{2}m(\alpha_{i+1}^2 - \alpha_i^2) + p(\alpha_{a+1} - \alpha_i) \\
            & \int_{\alpha_i}^{\alpha_{i+1}} f(\alpha_i)dx = (m\alpha_i + p) * (\alpha_{i+1} - \alpha_i)
    \end{aligned}\right.
\]

La méthode des rectangles à droite est donc d'ordre zéro. On retrouve ce résultat pour la méthode des rectangles à gauche par un calcul similaire. Intéressons nous maintenant à la méthode du point milieu. Calculons :
\[
    \begin{aligned} \int_{\alpha_i}^{\alpha_{i+1}} f\left(\frac{\alpha_i + \alpha_{i+1}}{2}\right) & = \left(m\frac{\alpha_i + \alpha_{i+1}}{2} + p\right) * (\alpha_{i+1} - \alpha_i) \\
                                                                                                   & = \frac{1}{2}m(\alpha_{i+1}^2 - \alpha_i^2) + p(\alpha_{i+1} - \alpha_i)
    \end{aligned}
\]

La méthode du point milieu est donc au moins d'ordre un. Des calculs avec des polynômes d'ordre deux permettraient de montrer que cette méthode est d'ordre exactement deux.

\subsection{Méthode des trapèzes}
On calcule l'aire du trapèze formé par les points $f(\alpha_i)$ et $f(\alpha_{i+1})$.

On utilise l'interpolation de Lagrange pour créer le polynôme de degré 1 passant par les deux point voulus : \[
    p_1(x) = f(\alpha_i)\frac{x - \alpha_{i+1}}{\alpha_i-\alpha_{i+1}} + f(\alpha_{i+1})\frac{x - \alpha_i}{\alpha_{i+1} - \alpha_i}
\]

L'approximation par la méthode des trapèzes donne : \[
    \int_{\alpha_i}^{\alpha_{i+1}} f(x)dx \approx \int_{\alpha_i}^{\alpha_{i+1}} p_1(x)dx = I_T
\]

On calcule : \[ \begin{aligned}
        I_T & = f(\alpha_i)\int_{\alpha_i}^{\alpha_{i+1}}\frac{x - \alpha_{i+1}}{\alpha_i-\alpha_{i+1}}dx + f(\alpha_{i+1})\int_{\alpha_i}^{\alpha_{i+1}}\frac{x - \alpha_i}{\alpha_{i+1} - \alpha_i}dx \\
            & = \frac{f(\alpha_i)}{\alpha_i - \alpha_{i+1}}\left[\frac{1}{2}(x - \alpha_{i+1})^2\right]_{\alpha_i}^{\alpha_{i+1}} + \frac{f(\alpha_{i+1})}{\alpha_{i+1} - \alpha_i}\left[\frac{1}{2}(x-\alpha_i)^2\right]_{\alpha_i}^{\alpha_{i+1}} \\
            & = (\alpha_{i+1} - \alpha_i)\frac{f(\alpha_i) + f(\alpha_{i+1})}{2}
    \end{aligned}
\]

On a donc l'approximation suivante : \begin{equation}
    \int_{\alpha_i}^{\alpha_{i+1}} f(x)dx \approx (\alpha_{i+1} - \alpha_i)\frac{f(\alpha_i) + f(\alpha_{i+1})}{2}
\end{equation}

On peut montrer que c'est aussi une méthode d'ordre un, malgré le fait qu'elle donne un résultat différent de la méthode du point milieu.

\subsection{Méthode de Simpson}
On approche la fonction sur $[\alpha_i, \alpha_{i+1}]$ par un polynôme de degré 2, en considérent les deux points aux extrémités et le point milieu. On nomme ce polynôme $p_2(x)$.

On pose : \[ I_S = \int_{\alpha_i}^{\alpha_{i+1}} p_2(x)dx \]

Afin de simplifier les calculs, on pose $u(x) = Ax+B$, et on effectue le changement de variable $f(x) = g \circ u(x)$. On veut $u(\alpha_i) = -1$ et $u(\alpha_{i+1}) = 1$. On trouve donc $A = \frac{2}{\alpha_{i+1} - \alpha_i}$ et $B = -\frac{\alpha_i + \alpha_{i+1}}{\alpha_{i+1} - \alpha_i}$

Effectuons le changement de variable :
\[ \int_{\alpha_i}^{\alpha_{i+1}} f(x)dx = \int_{\alpha_i}^{\alpha_{i+1}} g(u(x))dx = \frac{1}{A}\int_{-1}^1 g(u)du = \frac{\alpha_{i+1} - \alpha_{i}}{2} \int_{-1}^{1} g(u)du \]

On cherche maintenant à exprimer $\tld{p_2}$ tel que $p_2 = \tld{p_2} \circ u$, $p_2$ étant obtenu par la méthode d'interpolation de Lagrange. On a :
\[ \tld{p_2}(u) = \frac{1}{2}f(\alpha_i)u(u-1) + f\left(\frac{\alpha_i + \alpha_{i+1}}{2}\right)(u+1)(1-u) + \frac{1}{2}f(\alpha_{i+1})u(u+1) \]

On peut désormais calculer $I_S$ :
\[  \begin{aligned}
        I_S & = \int_{\alpha_i}^{\alpha_{i+1}} p_2(x)dx = \int_{-1}^1 \tld{p_2}(u)du \\
            & = \frac{1}{2}f(\alpha_i)\int_{-1}^1 (u^2-u)du + f\left(\frac{\alpha_i+\alpha_{i+1}}{2}\right)\int_{-1}^1 (1-u^2)du + \frac{1}{2}f(\alpha_{i+1})\int_{-1}^1 (u^2+u)du \\
            & = \frac{1}{2}f(\alpha_i)\left[\frac{1}{3}u^3-\frac{1}{2}u^2\right]_{-1}^1
                    + f\left(\frac{\alpha_i+\alpha_{i+1}}{2}\right)\left[u-\frac{1}{3}u^3\right]_{-1}^1
                    + \frac{1}{2}f(\alpha_{i+1})\left[\frac{1}{3}u^3+\frac{1}{2}u^2\right]_{-1}^1 \\
            & = \frac{1}{3}f(\alpha_i) + \frac{4}{3}f\left(\frac{\alpha_i + \alpha_{i+1}}{2}\right) + \frac{1}{3}f(\alpha_{i+1})
    \end{aligned}
\]

On obtient donc l'approximation suivante : \begin{equation}
    \int_{\alpha_i}^{\alpha_{i+1}} f(x)dx \approx \frac{1}{3}f(\alpha_i) + \frac{4}{3}f\left(\frac{\alpha_i + \alpha_{i+1}}{2}\right) + \frac{1}{3}f(\alpha_{i+1})
\end{equation}

\end{document}
