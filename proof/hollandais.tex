\section{Le drapeau hollandais}
\subsection{Objectif}
Le but de cet exercice est de trier en une seule passe un tableau contenant
seulement trois valeurs. Dans notre cas, ces trois valeurs seront $-1$, $0$ et
$1$, qui devront être classée dans cet ordre.

\subsection{Algorithme}
L'algorithme utilisé est le suivant : \newline
\begin{algorithm}[H]
    \SetAlgoLined
    \KwIn{a = tableau de valeurs}
    \KwData{m1 = 0}
    \KwData{m2 = taille de a moins 1}
    \KwData{i = 0}
    \While{i $\leq$ m2}{
        \eIf{a[i] = $-1$}{
            Échanger a[m1] et a[i]\;
            m1 = m1 + 1\;
            i = i + 1\;
        }{
            \eIf{a[i] = $1$} {
                Échanger a[m2] et a[i]\;
                m2 = m2 - 1\;
            }{
                i = i + 1\;
            }
        }
    }
\end{algorithm}

\subsection{Preuve}
Afin de prouver cet algorithme, nous devons trouver un invariant de boucle. Cet
invariant est l'état du tableau. En effet, nous allons considérer que le
tableau est toujours dans l'état suivant :

\begin{center}
    % TODO improve appearance
    \begin{tabular}{|ccc|ccc|ccc|ccc|}
        \hline
        ... & $-1$ & ... & ... & $0$ & ... & ... & $X$ & ... & .... & $1$ & ...\\
        \hline
            &&& m1 &&& i &&& m2 && \\
    \end{tabular}
\end{center}

Avec $X$ la zone des valeurs non encore triées.

\paragraph{Initialisation} À l'état initial, le tableau est considéré comme
étant intégralement dans l'état $X$. Les trois autres zones sont de taille
nulles (en effet, $m1$ et $i$ valent $0$ tandis que $m2$ est à la fin du
tableau).

\paragraph{Hérédité} On considère l'élément d'indice $i$. On traite trois cas :
\begin{itemize}
    \item $a[i] = 0$ Dans ce cas, la case d'indice $i$ fait déjà partie de la
        zone $0$, il suffit d'incrémenter $i$.
    \item $a[i] = -1$ Dans ce cas, il faut placer le contenu de la case dans la
        zone $-1$. Pour ce faire, on échange son contenu avec celui de la case
        d'indice $m1$ (fin de la zone de $-1$), puis on incrémente $-1$ pour
        indiquer que la taille de la zone a augmentée. La case d'indice $i$
        contient maintenant $0$ si la zone de zéro était non vide, $-1$ sinon.
        Dans les deux cas, le tableau est toujours dans le bon état : on peut
        incrémenter $i$.
    \item $a[i] = 1$ dans ce cas, on échange le contenu de la case avec celui
        de la case d'indice $m2$ puis on décrémente $m2$ afin d'indiquer que la
        taille de la zone de $1$ vient d'augmenter.
\end{itemize}
Dans chacun de ces trois cas, le tableau conserve son état, tandis que la
taille de la zone $X$ diminue.

\paragraph{Fin} L'algorithme se termine quand la taille de la zone $X$ est
nulle. À ce moment, le tableau est toujours dans l'état d'après l'hérédité, il
est donc trié.

